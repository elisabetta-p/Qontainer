\documentclass[a4paper,10pt]{article}
\usepackage[margin=1in, paperwidth=8.5in, paperheight=11in]{geometry}
\usepackage{amsfonts}
\usepackage{amssymb}
\usepackage[T1]{fontenc}
\usepackage{fancyvrb}
\usepackage[utf8]{inputenc}
\usepackage[english, italian]{babel}
\usepackage{tabularx}
\inputencoding{utf8}
\usepackage{float}
\usepackage{fancyhdr}
\usepackage{graphicx}
\pagestyle{fancy}
\renewcommand{\topfraction}{.9}
\renewcommand{\bottomfraction}{.9}
\renewcommand{\textfraction}{.1}
\fancyhf{}
\lfoot{Qontainer - Elisabetta Piombin - 1142189}


\begin{document}
\title {Relazione del progetto di Programmazione ad Oggetti: Qontainer}
\author {Elisabetta Piombin 1142189}
\date{}

\maketitle
\clearpage
\tableofcontents
\clearpage\mbox{}

\section{Abstract}
\textbf{Qontainer} è un progetto realizzzato al fine di fornire un contenitore che gestisca una libreria di contenuti multimediali: file audio e file video, che si dividono a loro volta in canzoni, podcast, serie tv e film. \\
Per farlo, la classe templatizzata \texttt{container} a sua volta fa uso di altre classi annidate:
\begin{enumerate}
\item \texttt{nodo}: inserita nella parte privata di \texttt{container}, viene usata per memorizzare i vari contenuti multimediali, visti come se fossero una lista concatenata di elementi, con ogni nodo diviso nel suo campo \texttt{info} (di tipo parametrico \texttt{T}, \texttt{next} e \texttt{prev}(di tipo \texttt{nodo*}). In \texttt{container} è presente un puntatore al primo elemento della lista.
\item \texttt{const\_iterator}: inserita nella parte pubblica di \texttt{container}, è la classe che permette l'implementazione di iteratori costanti.
\item \texttt{iterator}: inserita nella parte pubblica di \texttt{container}, è la classe che permette l'implementazione di iteratori non costanti.
\end{enumerate}
Vista l'assenza di puntatori ad altre classi nella gerarchia, è stata ritenuta superflua l'implementazione di una classe per un eventuale \textit{smart pointer}.\\
Ho scelto di implementare il contenitore come una lista linkata composta da nodi perché offro la possibilità di rimuovere contenuti all'interno della libreria in posizioni arbitrarie, e questo mi porta alla necessità di scegliere l'implementazione sotto forma di lista, che rende tale operazione di rimozione più efficiente (viene operata in tempo costante) rispetto alla scelta di implementare tale contenitore come un vettore di elementi.


\section{Compilazione ed esecuzione}
La compilazione si esegue tramite i comandi \texttt{qmake} e \texttt{make}. 

\section{Descrizione della gerarchia di classi}

La classe base astratta da cui deriva tutta la gerarchia è \texttt{ContenutoMultimediale}, che verrà poi concretizzata tramite sue classi derivate. I metodi virtuali sono \texttt{riproduci()}, \texttt{pausa()}, \texttt{operator==} e il relativo distruttore; \texttt{riproduci()} e \texttt{pausa()} sono anche puri, poiché la loro implementazione è assegnata alle classi derivate. \\
Da \texttt{ContenutoMultimediale} derivano immediatamente altre due classi: \texttt{audio} e \texttt{video}, che sono le due macrocategorie di appartenenza dei file che vengono memorizzati nella libreria. \\
Da \texttt{video} derivano due classi: \texttt{film} e \texttt{episodio}, mentre da \texttt{audio} derivano altre due classi, \texttt{podcast} e \texttt{canzone}. \\
Non si verifica la situazione di ereditarità multipla. \\


\section{Descrizione delle chiamate polimorfe}
Sono presenti tre metodi virtuali nella gerarchia, i metodi di \texttt{serializza(char)}, l'overloading dell'operatore di confronto \texttt{operator==(const T\&)}, e il metodo \texttt{qualita()}, che è puro.
\begin{enumerate}
	\item [-] \texttt{string serializza(char)}: definito virtuale all'interno di \texttt{contenutomultimediale}, viene implementato diversamente per i campi dati specifici di ogni classe derivata. Viene richiamata all'interno della classe \texttt{database}, all'interno della funzione \texttt{void save(container<T>\&)}, in quanto questa è la funzione che si occupa del salvataggio su file degli oggetti inseriti dentro al \texttt{container}.
	\item [-] \texttt{bool operator== (const T\&)} è usato nella funzione \texttt{void remove(T)} all'interno di \texttt{contenutomultimediale}, e viene usata in \texttt{risultatoricerca} (nella GUI) per rimuovere un elemento dal contenitore. 
	\item [-] \texttt{string qualita() const}, metodo virtuale puro in \texttt{contenutomultimediale}, utilizzato in \texttt{visualizzaelemento} per dare la possibilità all'utente per vedere se il file che si sta visualizzando è di qualità media, alta o bassa. Ha diverse implementazioni in \texttt{audio} e in \texttt{video}.
\end{enumerate}



\section{Descrizione del formato del file di salvataggio e caricamento del contenitore}
Per il caricamento dei file che compongono la libreria, e per il loro salvataggio, mi appoggio ad una classe esterna alla gerarchia, che ho chiamato \texttt{database}. Questa classe contiene i metodi \texttt{load()} e \texttt{save()}. La classe \texttt{database} non è un template perché dipende strettamente da \texttt{ContenutoMultimediale}. 
\begin{enumerate}
\item[-] \texttt{void load(container<ContenutoMultimediale*>\&)}, tramite un'operazione di deserializzazione implementata ad hoc in ogni classe derivata concreta della gerarchia, permette la lettura del contenuto in ogni file \texttt{.txt}. Esiste un unico file \texttt{.txt}, in cui sono scritti tutti gli oggetti che verranno caricati nel \texttt{container}.  Database si occupa di caricare questi oggetti nella libreria tramite \texttt{static film* film::deserializza(const vector<string>\&), static episodio* episodio::deserializza(const vector<string>\&), static podcast* podcast::deserializza(const vector<string>\&)} e \texttt{static canzone* canzone::deserializza(const vector<string>\&)}, in modo che possano essere gestiti dal contenitore templetizzato.
\item[-] \texttt{void save(container<ContenutoMultimediale*>\&)} si occupa di memorizzare sul file \texttt{.txt} i caratteri corrispondenti agli oggetti contenuti in \texttt{container}, invocando il metodo \texttt{serializza(char)} corretto, che è definito come virtuale.
\end{enumerate}

\section{Tempistiche}

\begin{center}
 \begin{tabular}{||c | c ||} 
 \hline
 \textbf{Ore} & \textbf{Fase}  \\ 
 \hline\hline
 2 & Analisi preliminare del problema \\ 
 \hline
 1 & Progettazione del modello \\
 \hline
 1 & Progettazione della GUI\\
 \hline
 10 & Apprendimento libreria Qt \\
 \hline
 30 & Codifica modello e GUI\\ 
 \hline
 10 & Debugging \\
 \hline
 5 & Testing\\
 \hline
\end{tabular}
\end{center}

\end{document}