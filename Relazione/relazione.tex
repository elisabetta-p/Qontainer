\documentclass[a4paper,10pt]{article}
\usepackage[margin=1in, paperwidth=8.5in, paperheight=11in]{geometry}
\usepackage{amsfonts}
\usepackage{amssymb}
\usepackage[T1]{fontenc}
\usepackage{fancyvrb}
\usepackage[utf8]{inputenc}
\usepackage[english, italian]{babel}
\usepackage{tabularx}
\inputencoding{utf8}


\begin{document}
\title {Relazione del progetto di Programmazione ad Oggetti: Qontainer}
\author {Elisabetta Piombin 1142189}
\date{}

\maketitle

\tableofcontents

\section{Abstract}
\textbf{Qontainer} è un progetto realizzzato al fine di fornire un contenitore che gestisca una libreria di contenuti multimediali: file audio e file video, che si dividono a loro volta in canzoni, podcast, serie tv e film. \\
Per farlo, la classe templatizzata \texttt{container} a sua volta fa uso di altre classi annidate:
\begin{enumerate}
\item \texttt{nodo}: inserita nella parte privata di \texttt{container}, viene usata per memorizzare i vari contenuti multimediali, visti come se fossero una lista concatenata di elementi, con ogni nodo diviso nel suo campo \texttt{info} (di tipo parametrico \texttt{T} e \texttt{next} (di tipo \texttt{nodo*}). In \texttt{container} è presente un puntatore al primo elemento della lista.
\item \texttt{const\_iterator}: inserita nella parte pubblica di \texttt{container}, è la classe che permette l'implementazione di iteratori costanti.
\item \texttt{iterator}: inserita nella parte pubblica di \texttt{container}, è la classe che permette l'implementazione di iteratori non costanti.
\end{enumerate}
Vista l'assenza di puntatori ad altre classi nella gerarchia, è stata ritenuta superflua l'implementazione di una classe per un eventuale \textit{smart pointer}.
%Le classi \texttt{const\_iterator} ed  \texttt{iterator} al loro interno contengono l'overloading dei seguenti operatori:  \texttt{operator==, operator!=, %operator++} (postfisso e prefisso) \texttt{operator---}(postfisso e prefisso) \texttt{}

\section{Descrizione della gerarchia di classi}
La classe base astratta da cui deriva tutta la gerarchia è \texttt{contenutomultimediale}, che verrà poi concretizzata tramite sue classi derivate. I metodi virtuali sono \texttt{riproduci()}, \texttt{pausa()}, \texttt{operator==} e il relativo distruttore; \texttt{riproduci()} e \texttt{pausa()} sono anche puri, poiché la loro implementazione è assegnata alle classi derivate. \\
Da \texttt{contenutomultimediale} derivano immediatamente altre due classi: \texttt{audio} e \texttt{video}, che sono le due macrocategorie di appartenenza dei file che vengono memorizzati nella libreria. \\
Da \texttt{video} derivano due classi: \texttt{film} e \texttt{episodio}, mentre da \texttt{audio} derivano altre due classi, \texttt{podcast} e \texttt{canzone}. \\
Non si verifica la situazione di ereditarità multipla. 

\end{document}